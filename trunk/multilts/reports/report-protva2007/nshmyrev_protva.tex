\author{Н. В. Шмырёв}

\city{Москва}

\projecttitle{VoxForge}

\projecturl{\url{http://voxforge.org}}

\title{Синтез и распознавание русской речи с открытыми исходными данными}

\maketitle

\begin{abstract}
В докладе будет рассмотрена деятельность по поддержке русского
языка в открытых программных пакетах CMU Sphinx и Festival. Основное
внимание будет уделено проблеме создания открытых речевых баз и
вариантам их использования.
\end{abstract}

Использование речи для коммуникации с персональными компьютерами и 
автоматическими системами очень перспективно и уже давно служит темой обширных
исследований. Бытует мнение, что пока область недостаточно
развита для коммерческого применения, на самом деле это не так.
Перечислим основные задачи и степень их решения:

\begin{itemize}
\item синтез речи. В довольно широком смысле эта проблема решена, речь 
синтезаторов почти неотличима от речи обычного человека. Проблемы
остаются только в области синтеза эмоциональной речи;
\item управление машиной с помощью команд. В целом, это решённая задача. 
Проблемы появляются только при наличии помех или необходимости
проявления интеллекта машиной;
\item диктовка связанного текста. Эта задача в общем случае не решена,
но значительные успехи в этой области имеются.
\end{itemize}

Нужно учитывать, что идеала тут достичь сложно, даже человек не идеален в
распознавании и синтезе. В некоторых областях машина распознаёт даже
лучше, например, приведём такую таблицу из \cite{SLP}

\begin{tabular}{|l|c|c|c|}
\hline
Область речи & Словарь & Человек & Компьютер \\
\hline
Набор цифр & 10 & 0.009\% & 0.72\% \\
\hline
Разговор по телефону & 2000 & 3.8\% & 36.7\% \\
\hline
Статьи из журнала  & 5000 & 0.9\% & 4.5\% \\
\hline
Текст без смысла & 20000 & 7.6\% & 4.4\% \\
\hline
\end{tabular}

В отличие от подхода с глубоким изучением особенностей речи, требующего
работы высококвалифицированных специалистов, в настоящее время
преимущественно накапливаются огромные базы данных речи и
обрабатываются методами статистики. Некоторое знание языка в данном
подходе необходимо, но требования значительно более слабые. Возникают
трудности совсем иного рода --- огромный размер баз данных. Современная
база для синтеза --- порядка 30 часов речи одного диктора, для
распознавания --- 140--150 часов речи 1000 человек.

Системы распознавания и синтеза с открытым кодом в том числе и от
Microsoft (проект HTK) доступны для использования и изучения.
Практикуются автоматические методы обучения (создание модели марковских
процессов, деревьев решений, взвешенных автоматов и нейронных сетей).

Например, создание системы распознавание состоит из следущих шагов.
Сначала изучается язык и строится его описание. По большому набору
текстов строится статистическая модель языка, создаётся фонетический
словарь. Например, точный русский фонетический словарь не существует, но
достаточно хорошее приближение можно построить с помощью небольшого
набора правил и доступного словаря ударений \cite{zaliznyak}. На
вход программы тренировки передаётся словарь, текст и озвучка этого
текста большим количеством дикторов. В результате получается модель,
которую можно использовать для распознавания слитной речи.

Таким образом, основная работа состоит в сборе больших баз данных,
ценность которых огромна, они позволяют:

\begin{itemize}
\item создавать синтезаторы и системы распознавания;
\item сопоставлять различные методы создания речевых систем;
\item изучать особенности языка;
\item свободные базы обладают ещё одним преимуществом --- их можно модифицировать и пополнять.
\end{itemize}

Перечислим проекты, посвящённые речевым технологиям:

\begin{itemize}
\item система для разработки синтезаторов Festival \cite{festival};
\item система распознавания речи CMU Sphinx \cite{sphinx};
\item ресурс VoxForge \cite{voxforge} посвящён сбору речевых баз.
Текущая база русского языка содержит 10 дикторов по 200 предложений, 2.5
часа речи. На тестовом наборе данных точность распознавания 80\%.
Английская база --- 15 часов речи;
\item синтез русской речи --- проект FestLang \cite{festlang},
посвящённый и синтезу русской речи. Предлагается база данных для синтеза
речи и готовый синтезатор.
\end{itemize}

В настоящий момент работа идёт по следущим направлениям:

\begin{itemize}
\item разработка DSP алгоритма с хорошими возможностями моделирования;
\item интонационная база русского языка для синтезатора;
\item сбор базы для распознавания слитной русской речи;
\item диалоговая система по управлению рабочим столом GNOME, в рамках Google SoC развивается проект gnome-voice-control.
\end{itemize}

\begin{thebibliography}{99}
\bibitem{SLP} Huang, Xuedong. Spoken Language Processing. Prentice Hall PTR, 2001.
\bibitem{festival} Festival TTS, \url{http://festvox.org}
\bibitem{sphinx} CMU Sphinx, \url{http://cmusphinx.sourceforge.net}
\bibitem{voxforge} VoxForge GPL Speech Corpora, \url{http://voxforge.org}
\bibitem{festlang} Festival Voices Development, \url{http://festlang.berlios.de}
\bibitem{zaliznyak} А. А. Зализняк. Грамматический словарь русского языка.
\end{thebibliography}

%%% Local Variables: 
%%% mode: latex
%%% TeX-master: "../main"
%%% End: 
