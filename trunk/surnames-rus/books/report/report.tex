\documentclass{seminar}

\usepackage[utf8]{inputenc}
\usepackage[T2A]{fontenc}
\usepackage[russian]{babel}
\usepackage{graphicx}
\usepackage{sem-a4}

\pagestyle{empty}

\begin{document}

\slideframe{none}
\sffamily


\begin{slide}

\center{\large Методы статистического моделирования речи в системе Festival}
\center{Николай Шмырёв}
\center{http://festvox.org}

\end{slide}

\begin{slide}

{\bf Festival -- среда для создания синтезаторов речи}

Разрабатывается начиная с 1996 года
\begin{itemize}
\item Alan W Black (CMU)
\item Rob Clark (CSTR)
\item Korin Richmond (CSTR)
\item Volker Strom (CSTR)
\item Simon King (CSTR)
\item Heiga Zen (Nagoya Institute of Technology)
\item Paul Taylor (Cambridge)
\item и многие другие...
\end{itemize}

Свободно распространяется, работает под Windows и Linux. Поддерживает
стандартные интерфейсы - Microsoft SAPI, SSML.

\end{slide}

\begin{slide}
\textbf{Составляющие:}
\begin{itemize}
\item SpeechTools - библиотека цифровой обработки сигналов, различные вспомогательные средства, средства статистической обработки, интерпретатор языка Scheme
\item Festival - система синтеза речи
\item Festvox - средства для построения голосов для festival, разметки речевых баз, преобразования голоса
\item CmuDict, Poslex - словари английского языка
\item Базы данных arctic - 5 дикторов по часу речи каждый. ked TIMIT - 
вручную размеченная база из 450 предложений. kal/ked дифоны.
\item Различные голоса для festival
\end{itemize}

\end{slide}

\begin{slide}
\textbf{Языки:}

\begin{tabular}{ll}
Валлийский & Вьетнамский \\
Датский & Испанский  \\
Каталанский & Немецкий \\
Польский & Португальский \\
Русский & Суахили/Ибибио \\
Финский & Французский \\
Хинди/Синхала/Телугу & Чешский \\
Шведский & Японский
\end{tabular}
\end{slide}

\begin{slide}
{\bf Цели системы}
\begin{itemize}
\item Образовательная
\item Стандартизующая (Blizzard challenge)
\item Исследовательская (AT\&T Crystal)
\item Производственная (WHISPR project)
\end{itemize}
\end{slide}                    

\begin{slide}
{\bf Этапы синтеза речи и их возможные реализации.}

Подготовка базы:
\begin{itemize}
\item Подбор корпуса для записи речевой базы: использование
      частотных анализаторов текста, подбор фонетически сбалансированных
      предлоежений
\item Сегментация записанной речи: сегментация по образцу, автоматическая
      сегментация с помощью HMM (ehmm и sphinx).
\item Вычисление акустических параметров и создание базы. 
\end{itemize}

Синтез:
\begin{itemize}
\item Разбиение на слова
\item Анализ структуры предложения, выделение содержательных слов - ngrams, Viterbi поиск, WFST
\item Преобразование в последовательность звуков, разбиение на слоги, выделение
      ударения: LTS правила, деревья решений для разбиения на слоги, LTS деревья, 
      деревья для редукции и предсказания ударения.
\item Разметка интонации: ToBI, tilt, ToDI - правила и деревья
\item Предсказание контура основного тона - деревья, правила
\item Предсказание длительности сегментов - деревья решений, правила, моделирование HMM, моделирования логарифма длительности
\item Синтез
\end{itemize}

\end{slide}
\begin{slide}

{\bf Методы синтеза}

\begin{itemize}
\item Дифоны
\item Выборка из базы с различными методами склейки - clunits, multisyn
\item Статистическое моделирование параметров - clustergen, htk
\item Моделирование речевого тракта
\end{itemize}    

\end{slide}

\begin{slide}

\begin{figure}
\begin{center}
\includegraphics[width=.9\textwidth]{./images/utterance.eps}
\end{center}
\end{figure}

\end{slide}
\begin{slide}
    
{\bf Модели, описывающие речевые процессы.}

Способы описания

\begin{itemize}
\item cоздание правил
\item создание словаря
\item создание модели
\end{itemize}

Модели

\begin{itemize}
\item Ngrams
\item CART деревья
\item WFST, скрытые марковские модели (HMM), поиск Viterbi
\end{itemize}

Немного в стороне, но из той же области - нейронные сети.
\end{slide}
\begin{slide}
Триграммы

\begin{tabular}{|c|c|c|c|}
\hline
a & b & d & 0.6 \\
\hline
a & d & c & 0.3 \\
\hline
d & b & c & 0.1 \\
\hline
c & a & a & 0.7 \\
\hline
\end{tabular}

\end{slide}
\begin{slide}
Марковская цепь

\begin{figure}
\begin{center}
\includegraphics[width=.9\textwidth]{./images/hmm.eps}
\end{center}
\end{figure}

\end{slide}
\begin{slide}
Дерево решений CART

\begin{figure}
\begin{center}
\includegraphics[width=.9\textwidth]{./images/cart.eps}
\end{center}
\end{figure}

\end{slide}
\begin{slide}
Поиск Витерби

\begin{figure}
\begin{center}
\includegraphics[width=.9\textwidth]{./images/viterbi.eps}
\end{center}
\end{figure}

\end{slide}

\begin{slide}

{\bf Тренировка дерева решений на примере дерева для расстановки ударений в русских фамилиях.}

Ariadna Font Llitjós, Improving Pronunciation Accuracy of Proper Names with Language Origin Classes, Master Thesis 2003

\end{slide}
\begin{slide}

{\bf Образец вектора признаков и результатов тренировки}
{
\small
\begin{verbatim}
(o b r a z c o v a)
0 4 2 o 3 1 + pau pau 0 0 0 0 0 0 b - 0 + s l - r a z surname-ov z c o v a
0 3 3 a 2 1 - b r - 0 + l a - z - 0 0 f d - c o v surname-ov z c o v a
1 2 4 o 3 1 + z c - 0 - a d - v - 0 0 f b - a nil nil surname-ov z c o v a
0 1 5 a 2 1 - o v - 0 + f b - pau - 0 0 0 0 0 nil nil nil surname-ov z c o v a

FEATURE    1 p.ph_vfront: 0.7133
FEATURE    2 last.name: 0.7165
FEATURE    3 num2end: 0.8170
FEATURE    4 lastt.name: 0.8402
FEATURE    5 lasttttt.name: 0.8593

      0 3146  161 3307      [3146/3307]      95.132
      1  226 1123 1349      [1123/1349]      83.247
        3372 1284
      total 4656 correct 4269.000 91.688%
\end{verbatim}
}

\end{slide}


\begin{slide}
{\bf Образец дерева решений}
{\small
\begin{verbatim}
(set! english_phrase_type_tree
'((pbreak is NB)
  ((num_break is 1)
   ((mB))
   ((R:Token.parent.EMPH is 1)
    ((NB))
    ((n.R:Token.parent.EMPH is 1)
     ((NB))
     ((NB)))))
  ((pbreak is BB)
   ((BB))
   ((pbreak is mB)
    ((mB))
    ((name in ("." "!" "?"));; only (potentially) change Bs to BBs
     ((BB))
     ((B)))))))
\end{verbatim}
}
\end{slide}


\begin{slide}
{\bf Оптимальный набор параметров:}

\begin{itemize}
\item Число слогов до конца слова
\item Число слогов до начала слова
\item Предыдущий, предсказываемый, три следующих сегмента
\item Три сегмента перед концом слова
\end{itemize}

Точность - 92.5 на независимых данных

\end{slide}

\begin{slide}

{\bf Спасибо за внимание!}

Пишите:

\begin{itemize}
\item \textit{nshmyrev@yandex.ru}
\item \textit{festival-talk@festvox.org} {\small (необходимо сначала подписаться на список рассылки)}
\end{itemize}

Заходите:

\begin{itemize}
\item \textit{http://festvox.org}
\item \textit{http://festlang.berlios.de}
\end{itemize}

\end{slide}
\end{document}
